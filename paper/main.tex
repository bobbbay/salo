\documentclass[conference]{IEEEtran}

\usepackage{ebproof} % http://ftp.tu-chemnitz.de/pub/tex/macros/latex/contrib/ebproof/ebproof.pdf
\usepackage{listings}
\usepackage{amssymb}

\usepackage{cite}      % Written by Donald Arseneau
                        % V1.6 and later of IEEEtran pre-defines the format
                        % of the cite.sty package \cite{} output to follow
                        % that of IEEE. Loading the cite package will
                        % result in citation numbers being automatically
                        % sorted and properly "ranged". i.e.,
                        % [1], [9], [2], [7], [5], [6]
                        % (without using cite.sty)
                        % will become:
                        % [1], [2], [5]--[7], [9] (using cite.sty)
                        % cite.sty's \cite will automatically add leading
                        % space, if needed. Use cite.sty's noadjust option
                        % (cite.sty V3.8 and later) if you want to turn this
                        % off. cite.sty is already installed on most LaTeX
                        % systems. The latest version can be obtained at:
                        % http://www.ctan.org/tex-archive/macros/latex/contrib/supported/cite/

%\usepackage{graphicx}  % Written by David Carlisle and Sebastian Rahtz
                        % Required if you want graphics, photos, etc.
                        % graphicx.sty is already installed on most LaTeX
                        % systems. The latest version and documentation can
                        % be obtained at:
                        % http://www.ctan.org/tex-archive/macros/latex/required/graphics/
                        % Another good source of documentation is "Using
                        % Imported Graphics in LaTeX2e" by Keith Reckdahl
                        % which can be found as esplatex.ps and epslatex.pdf
                        % at: http://www.ctan.org/tex-archive/info/
\input epsf
\usepackage{graphicx}
%+++++++++++++++++++++++++++++++++++++++++++
% However, be warned that pdflatex will require graphics to be in PDF
% (not EPS) format and will preclude the use of PostScript based LaTeX
% packages such as psfrag.sty and pstricks.sty. IEEE conferences typically
% allow PDF graphics (and hence pdfLaTeX). However, IEEE journals do not
% (yet) allow image formats other than EPS or TIFF. Therefore, authors of
% journal papers should use traditional LaTeX with EPS graphics.
%
% The path(s) to the graphics files can also be declared: e.g.,
% \graphicspath{{../eps/}{../ps/}}
% if the graphics files are not located in the same directory as the
% .tex file. This can be done in each branch of the conditional above
% (after graphicx is loaded) to handle the EPS and PDF cases separately.
% In this way, full path information will not have to be specified in
% each \includegraphics command.
%
% Note that, when switching from latex to pdflatex and vice-versa, the new
% compiler will have to be run twice to clear some warnings.


%\usepackage{psfrag}    % Written by Craig Barratt, Michael C. Grant,
                        % and David Carlisle
                        % This package allows you to substitute LaTeX
                        % commands for text in imported EPS graphic files.
                        % In this way, LaTeX symbols can be placed into
                        % graphics that have been generated by other
                        % applications. You must use latex->dvips->ps2pdf
                        % workflow (not direct pdf output from pdflatex) if
                        % you wish to use this capability because it works
                        % via some PostScript tricks. Alternatively, the
                        % graphics could be processed as separate files via
                        % psfrag and dvips, then converted to PDF for
                        % inclusion in the main file which uses pdflatex.
                        % Docs are in "The PSfrag System" by Michael C. Grant
                        % and David Carlisle. There is also some information 
                        % about using psfrag in "Using Imported Graphics in
                        % LaTeX2e" by Keith Reckdahl which documents the
                        % graphicx package (see above). The psfrag package
                        % and documentation can be obtained at:
                        % http://www.ctan.org/tex-archive/macros/latex/contrib/supported/psfrag/

%\usepackage{subfigure} % Written by Steven Douglas Cochran
                        % This package makes it easy to put subfigures
                        % in your figures. i.e., "figure 1a and 1b"
                        % Docs are in "Using Imported Graphics in LaTeX2e"
                        % by Keith Reckdahl which also documents the graphicx
                        % package (see above). subfigure.sty is already
                        % installed on most LaTeX systems. The latest version
                        % and documentation can be obtained at:
                        % http://www.ctan.org/tex-archive/macros/latex/contrib/supported/subfigure/

\usepackage{url}       % Written by Donald Arseneau
                        % Provides better support for handling and breaking
                        % URLs. url.sty is already installed on most LaTeX
                        % systems. The latest version can be obtained at:
                        % http://www.ctan.org/tex-archive/macros/latex/contrib/other/misc/
                        % Read the url.sty source comments for usage information.

%\usepackage{stfloats}  % Written by Sigitas Tolusis
                        % Gives LaTeX2e the ability to do double column
                        % floats at the bottom of the page as well as the top.
                        % (e.g., "\begin{figure*}[!b]" is not normally
                        % possible in LaTeX2e). This is an invasive package
                        % which rewrites many portions of the LaTeX2e output
                        % routines. It may not work with other packages that
                        % modify the LaTeX2e output routine and/or with other
                        % versions of LaTeX. The latest version and
                        % documentation can be obtained at:
                        % http://www.ctan.org/tex-archive/macros/latex/contrib/supported/sttools/
                        % Documentation is contained in the stfloats.sty
                        % comments as well as in the presfull.pdf file.
                        % Do not use the stfloats baselinefloat ability as
                        % IEEE does not allow \baselineskip to stretch.
                        % Authors submitting work to the IEEE should note
                        % that IEEE rarely uses double column equations and
                        % that authors should try to avoid such use.
                        % Do not be tempted to use the cuted.sty or
                        % midfloat.sty package (by the same author) as IEEE
                        % does not format its papers in such ways.

\usepackage{amsmath}   % From the American Mathematical Society
                        % A popular package that provides many helpful commands
                        % for dealing with mathematics. Note that the AMSmath
                        % package sets \interdisplaylinepenalty to 10000 thus
                        % preventing page breaks from occurring within multiline
                        % equations. Use:
\interdisplaylinepenalty=2500
                        % after loading amsmath to restore such page breaks
                        % as IEEEtran.cls normally does. amsmath.sty is already
                        % installed on most LaTeX systems. The latest version
                        % and documentation can be obtained at:
                        % http://www.ctan.org/tex-archive/macros/latex/required/amslatex/math/



% Other popular packages for formatting tables and equations include:

%\usepackage{array}
% Frank Mittelbach's and David Carlisle's array.sty which improves the
% LaTeX2e array and tabular environments to provide better appearances and
% additional user controls. array.sty is already installed on most systems.
% The latest version and documentation can be obtained at:
% http://www.ctan.org/tex-archive/macros/latex/required/tools/

% Mark Wooding's extremely powerful MDW tools, especially mdwmath.sty and
% mdwtab.sty which are used to format equations and tables, respectively.
% The MDWtools set is already installed on most LaTeX systems. The lastest
% version and documentation is available at:
% http://www.ctan.org/tex-archive/macros/latex/contrib/supported/mdwtools/


% V1.6 of IEEEtran contains the IEEEeqnarray family of commands that can
% be used to generate multiline equations as well as matrices, tables, etc.


% Also of notable interest:

% Scott Pakin's eqparbox package for creating (automatically sized) equal
% width boxes. Available:
% http://www.ctan.org/tex-archive/macros/latex/contrib/supported/eqparbox/

% Correct bad hyphentation here
\hyphenation{op-tical net-works semi-conduc-tor IEEEtran}
\begin{document}

\title{\LARGE Declarative platform-agnostic dependently typed build and deployment configuration languages for operating systems}
\author{SEMC, Nexus Aurora}

\maketitle

\begin{abstract}
In the modern computing industry, operating systems are one of the most widely-configured parts of a project. Their management of numerous key computers components - such as  input/output, memory, and packages - makes them a beehive for never-ending configuration. Without a unified system, information about current configuration can be lost over time. New members looking at a deployed OS are very unlikely to catch every modification at first glance, software updates are likely to introduce new configuration requirements, and making even small changes in general can be a complete mess. This paper introduces Salo: a declarative, dependently-typed configuration language that eliminates this issue entirely. With a language such as Salo, changes to configuration are centralized in one set of declarations, solving all of the previously-mentioned issues with operating system maintainership - and more.
\end{abstract}
\IEEEoverridecommandlockouts
\begin{keywords}
Language design, dependent types, configuration languages, type theory.
\end{keywords}

\IEEEpeerreviewmaketitle

\section{Introduction}

TODO: Intro

\section{Language}

\subsection{Syntax}

TODO: Syntax

\subsection{Type System}

\newcommand{\type}{\textrm{Type}}

Naming conventions:

\begin{itemize}
    \item Lowercase Latin letters (e.g. $x$, $y$): variables
    \item Capital Latin letters (e.g. $A$, $B$): types
    \item Capital Greek letters (e.g. $\Gamma$, $\Delta$): contexts
\end{itemize}

\vspace{10pt}

\subsubsection{Zero type}

\[
\begin{prooftree}
    \hypo { }
    \infer1 { \vdash \varnothing : \type }
\end{prooftree}
\]

\[
\begin{prooftree}
    \hypo { \Gamma \vdash p : \varnothing, \quad \Gamma \vdash C : \type }
    \infer1 { \Gamma \vdash \textrm{abort}(p) : C }
\end{prooftree}
\]

\subsubsection{Unit type}

\[
\begin{prooftree}
    \hypo { }
    \infer1 { \vdash \textrm{unit} : \type }
\end{prooftree}
, \quad
\begin{prooftree}
    \hypo { }
    \infer1 { \vdash \textrm{()} : \textrm{unit} }
\end{prooftree}
\]

\[
\begin{prooftree}
    \hypo { \Gamma \vdash C : \type, \quad \Gamma \vdash c : C, \quad \Gamma \vdash p : \textrm{unit} }
    \infer1 { \Gamma \vdash \textrm{triv}(p, c) : C }
\end{prooftree}
\]

\[
\begin{prooftree}
    \hypo { \Gamma \vdash C : \type, \quad \Gamma \vdash c : C }
    \infer1 { \Gamma \vdash \textrm{triv}(\textrm{tt}, c) \equiv c : C }
\end{prooftree}
\]

\subsubsection{Product types}

\[
\begin{prooftree}
    \hypo { \Gamma \vdash A : \type, \quad \Gamma \vdash B : \type }
    \infer1 { \Gamma \vdash A \times B : \type }
\end{prooftree}
\]

\[
\begin{prooftree}
    \hypo { \Gamma \vdash a : A, \quad \Gamma \vdash b : B }
    \infer1 { \Gamma \vdash (a, b) : A \times B }
\end{prooftree}
\]

\[
\begin{prooftree}
    \hypo { \Gamma \vdash C : \type, 
            \quad \Gamma \vdash p : A \times B,
            \quad \Gamma, x : A, y : B \vdash c : C }
    \infer1 { \Gamma \vdash \textrm{unpack}(p, c) : C }
\end{prooftree}
\]

\[
\begin{prooftree}
    \hypo { \Gamma \vdash C : \type, 
            \quad \Gamma \vdash a : A, }
    \infer[no rule]1 {
            \quad \Gamma \vdash b : B,
            \quad \Gamma, x : A, y : B \vdash c : C }
    \infer1 { \Gamma \vdash \textrm{unpack}((a, b), c) \equiv c[a/x, b/y] : C }
\end{prooftree}
\]

\subsubsection{Sum types}

\[
\begin{prooftree}
    \hypo { \Gamma \vdash A : \type, 
            \quad \Gamma \vdash B : \type }
    \infer1 { \Gamma \vdash A + B : \type }
\end{prooftree}
\]

\[
\begin{prooftree}
    \hypo { \Gamma \vdash a : A, 
            \quad \Gamma \vdash B : \type }
    \infer1 { \Gamma \vdash \textrm{inl}(a) : A + B }
\end{prooftree}
\quad , \quad
\begin{prooftree}
    \hypo { \Gamma \vdash A : \type,
            \quad \Gamma \vdash b : B }
    \infer1 { \Gamma \vdash \textrm{inr}(b) : A + B }
\end{prooftree}
\]

\vspace{5pt}

\[
\begin{prooftree}
    \hypo { \Gamma \vdash C : \type,
            \quad \Gamma \vdash p : A + B, }
    \infer[no rule]1 {
            \Gamma, x : A \vdash c_A : C,
            \quad \Gamma, y : B \vdash c_B : C }
    \infer1 { \Gamma \vdash \textrm{case}(p, c_A, c_B) : C }
\end{prooftree}
\]

\vspace{5pt}

\[
\begin{prooftree}
    \hypo { \Gamma \vdash C : \type,
            \quad \Gamma \vdash a : A, }
    \infer[no rule]1 {
            \Gamma, x : A \vdash c_A : C,
            \quad \Gamma, y : B \vdash c_B : C }
    \infer1 { \Gamma \vdash \textrm{case}(\textrm{inl}(a), c_A, c_B) \equiv c_A[a/x] : C }
\end{prooftree}
\]

\vspace{5pt}

\[
\begin{prooftree}
    \hypo { \Gamma \vdash C : \type,
            \quad \Gamma \vdash b : B, }
    \infer[no rule]1 {
            \Gamma, x : B \vdash c_A : C,
            \quad \Gamma, y : A \vdash c_B : C }
    \infer1 { \Gamma \vdash \textrm{case}(\textrm{inr}(b), c_A, c_B) \equiv c_A[a/x] : C }
\end{prooftree}
\]

\subsubsection{Anonymous function types}

\[
\begin{prooftree}
    \hypo { \Gamma \vdash A : \type,
            \quad \Gamma \vdash B : \type, }
    \infer1 { \Gamma \vdash A \rightarrow B : \type }
\end{prooftree}
\]

\[
\begin{prooftree}
    \hypo { \Gamma, x : A \vdash y : B }
    \infer1 { \Gamma \vdash \lambda(x)(y) : A \rightarrow B }
\end{prooftree}
\]

\[
\begin{prooftree}
    \hypo { \Gamma \vdash f : A \rightarrow B,
            \quad \Gamma \vdash a : A }
    \infer1 { \Gamma \vdash f \hspace{3pt} a : B }
\end{prooftree}
\]

\[
\begin{prooftree}
    \hypo { \Gamma, x : A \vdash b : B,
            \quad \Gamma \vdash a : A }
    \infer1 { \Gamma \vdash \lambda(x)(b) a \equiv b[a/x] : B }
\end{prooftree}
\]

\subsection{Standard Library}

TODO: Standard library

\section{Operating System Building}

TODO: Building an OS.

\section{Operating System Deployment}

TODO: Deploying an OS.

\section{Conclusion}

TODO: Conclusion

\section*{Acknowledgment}
\addcontentsline{toc}{section}{Acknowledgment}

TODO: Acknowledgment

\begin{thebibliography}{1}

\bibitem{dolstra} E. Dolstra, M. de Jonge, and E. Visser, \emph{Nix: A Safe and Policy-Free System for Software Deployment}, Utrecht, Netherlands: Utrecht University, November 14-19 2004.

%\bibitem {cantrell1}
%W. H. Cantrell, ``Tuning analysis for the high-Q class-E power
%amplifier,'' \emph{IEEE Trans. Microwave Theory \& Tech.}, vol. 48,
%no. 12, pp. 2397-2402, December 2000.

%\bibitem {cantrell2}
%W. H. Cantrell, and W. A. Davis, ``Amplitude modulator utilizing a
%high-Q class-E DC-DC converter'', \emph {2003 IEEE MTT-S Int. Microwave
%Symp. Dig.}, vol. 3, pp. 1721-1724, June 2003.

%\bibitem {krauss}
%H. L. Krauss, C. W. Bostian, and F. H. Raab, \emph{Solid State Radio Engineering}, New York: J. Wiley \& Sons, 1980.

%\bibitem{IEEEhowto:kopka}
%H.~Kopka and P.~W. Daly, \emph{A Guide to {\LaTeX}}, 3rd~ed.\hskip 1em plus
% 0.5em minus 0.4em\relax Harlow, England: Addison-Wesley, 1999.

%\bibitem{lamport} L. Lamport, \emph{ {\LaTeX} A Document Preparation
%  System}, Reading, Mass: Addison-Wesley, 1994.

%\bibitem{knuth} D. E. Knuth, \emph {The \TeX book}, Reading, Mass.:
%  Addison-Wesley, 1996.

\end{thebibliography}

% That's all, folks!
\end{document}
